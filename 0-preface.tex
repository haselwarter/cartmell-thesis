
A notion of equational theory is introduced; more general than previous notions, equal in decsriptive power to the essentially algebraic theories of Freyd \cite{freyd:aspects-of-topoi}, and hence to the logic of left exact categories, we call the theories generalised algebraic.
%
The extra generality of these equational theories is achieved by the introduction of sort structures more general than those usually considered in that sorts may denote sets as is usual or else they may denote families of sets, families of families of sets and the like.
%
This acceptance of variable types at the level of syntax (the idea and the form of syntax is taken directly from Martin-Löf type theory) makes the theories particularly suited to the description of the structures that occur in category theory.
%
The basic example being the theory of categories, in which $\synOb$ appears as a sort to be interpreted as a set whereas $\synHom$ appears as a sort to be interpreted as a family of sets indexed by $\synOb \times \synOb$.
%
$\synHom(x,y)$ appears in the syntax as a variable type.

The definition of the most general or algebraic semantics for generalised algebraic theories necessitates the introduction of the notion of a contextual category.
%
So called because we shall see that the objects of a contextual category should be thought of as contexts.

The theory of contextual categories is seen as an algebraic description of the structure imposed on certain classes of term and type expressions by the operation of substitution of correctly typed term variables.
%
Now this is something one might also say of the theory of categories.
%
However the theory of contextual categories captures the structure of substitution at work in a more general situation, it is the structure of substitution as found in the generalised algebraic theories but not in algebraic theories, as found originally in Martin-Löf type theory but not in theories of the typed $\lambda$-calculus.

It is proved that the category of contextual categories is equivalent to the category of generalised algebraic theories and equivalence classes of interpretations.
%
Thus we say that we have the most general possible semantics.
%
This result is a generalisation of the result implicit in Lawvere \cite{lawvere:11} that the old syntactic notion of algebraic theory (i.e.\ one sorted equational) and Lawvere's algebraic notion are both one and the same (i.e.\ equivalent categories).

This thesis developed  from a desire to develop the model theory of Martin-Löf type theory.
%
The model theory rests on the notions of generalised algebraic theory and contextual category.
%
It is only in these terms that we can define the notion model of Martin-Löf type theory.
%
We also give the definition of model for a strengthened version of Martin-Löf type theory, this definition can be reworked algebraically into a hyperdoctrinal format.
%
We briefly describe a new model of the type theory in which types are interpreted as limit spaces.

The model theory of the strengthened version of Martin-Löf type theory is a generalisation of the well known correspondence of the typed $\lambda$-calculus with cartesian closed categories.

%%% Local Variables:
%%% mode: latex
%%% TeX-master: "cartmell-thesis"
%%% End: 