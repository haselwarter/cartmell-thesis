% source p1.1

The purpose of this chapter is to describe and to formally define the notion of generalised algebraic theory.
%
It is hoped that it will be clear from the description that (i) the notion is a natural one formalising actual mathematical language and that (ii) the notion is a simple generalisation of the notion of a many sorted algebraic theory.
%
\comment{TODO: format inline list!}
%
Though (ii) tends to be obscured by the form of the chosen syntax no doubt the choice is correce.

The formal definition is given in \textsection \ref{sec:source-1-6}.
%
Most of the material that follows \textsection \ref{sec:source-1-6} is in preparation for Chapter Two, \textsection \ref{sec:source-1-8} is partially in digression and partially to explain some of the informal syntax that is used in the early sections of this Chapter.

% source p1.2
\section{Introduction} \label{sec:source-1-1}

The notion of generalised algebraic theory is a generalisation of the notion of many sorted algebraic theory in just the following manner.
%
Whereas the sorts of a many sorted algebraic theory are constant types in the sense that they are to be interpreted as sets the sorts of a generalised algebraic theory need not all be constant types some of them may be nominated to be variable types in which case they are to be interpreted as families of sets.
%
The type or types on which the variation of a variable type depends must always be specified.

Thus a generalised algebraic theory consists of (i) a set of sorts, each with a specified role either as a constant type or else as a variable type varying in some way, (ii) a set of operator symbols, each with its argument types and its value type specified (the value type may vary as the argument varies), (iii) a set of axioms.
%
\comment{TODO: format inline list?}
%
Each axiom must be an identity beteen similar well formed expressions, either between terms of the same possibly varying type or else between type expressions.

The theory of categories is a good example.
%
The sort symbols we shall call $\synOb$ and $\synHom$, the operator symbols $\synid$ and $\syno$.

$\synOb$ is a constant type.  $\synHom$ is a symbol for a variable type depending twice on $\synOb$.
%
That is to say that if $t_1$ and $t_2$ are both terms of type $\synOb$ then $\synHom(t_1,t_2)$ is a type.
%
In particular if $x$ and $y$ are both variables of type $\synOb$ then $\synHom(x,y)$ is a type.

% source p1.3

The operator symbol $\synid$ has one argument type, namely $\synOb$.
%
The value type of $\synid$ varies as the argument varies, for if $x$ is a variable of type $\synOb$ then $\synid(x)$ is of type $\synHom(x,x)$.

Not all the argument types of $\syno$ are constant.
%
If $x$, $y$ and $x$ are variables of type $\synOb$, if $f$ is a variable of type $\synHom(x,y)$ and if $g$ is a variable of type $\synHom(y,z)$, then $\syno(f,g)$ is a term of type $\synHom(x,z)$.

One way of setting up the syntax to deal with variables would be to assume that for every type $\Delta$ we had a supply $\varsV_\Delta$ of variables of type $\Delta$.
%
However this method would lead to complications.
%
Instead we assume just one set $\varsV$ of variables and then repeatedly assign types to variables as required.
%
In a particular context the assertion or assumption that the variable $x$ is of type $\Delta$ is written shorthand as $x \in \Delta$.
%
More generally, the assertion that an expression $t$ is a term of type $\Delta$ will be written as $t \in \Delta$.
%
If the term $t$ has variables $x_1,\ldots x_n$ occurring within it then it will only make sense to assert $t \in \Delta$ under an assumption that $x_1,\ldots x_n$ are variables of particular types.
%
The complete assertion will be of the form: if $x_1$ is a variable of type $\Delta_1$, \ldots\ and if $x_n$ is a variable of type $\Delta_n$ then $t$ is a term of type $\Delta$.
%
This complete assertion we write shorthand as
%
$ \inferrule
  { x_1 \in \Delta_1, x_2 \in \Delta_2, \ldots\ x_n \in \Delta_n }
  { t \in \Delta } $
%
or else as $x_1 \in \Delta_1, x_2 \in \Delta_2, \ldots\ x_n \in \Delta_n : t \in \Delta$.
%
% source p1.4
%
Similarly 
%
$ \inferrule
  { x_1 \in \Delta_1, x_2 \in \Delta_2, \ldots\ x_n \in \Delta_n }
  { \Delta \isatype } $
%
is used to assert that if $x_1$ is a variable of type $\Delta_1$, \ldots\ if $x_n$ is a variable of type $\Delta_n$, then $\Delta$ is a type.

These shorthands of the forms 
%
$\inferrule
  { x_1 \in \Delta_1, x_2 \in \Delta_2, \ldots\ x_n \in \Delta_n }
  { t \in \Delta } $
%
and 
%
$\inferrule
  { x_1 \in \Delta_1, x_2 \in \Delta_2, \ldots\ x_n \in \Delta_n }
  { \Delta \isatype } $
%
we call \defemph{rules}.
%
They serve to express which expressions of a given language are well formed as terms or as types.
%
We work with these rules as units rather than with the basic expressions.
%
For example, in the formal definition instead of defining the notions of well formed term and well formed type we define inductively a set of rules, to be called the derivable rules, which express the well formed types, the well formed terms their types.

The axioms of a theory are also written as rules.
%
Instead of the more usual $\forall x_1 \in \Delta_1, \forall x_2 \in \Delta_2, \ldots \forall x_n \in \Delta_n, t_1 = t_2$ we write
%
$\inferrule
  { x_1 \in \Delta_1, x_2 \in \Delta_2, \ldots\ x_n \in \Delta_n }
  { t_1 = t_2 } $
%
There again, we might just write $t_1 = t_2$, whenever $x_1 \in \Delta_1, \ldots x_n \in \Delta_n$.

For example, the theory of categories has as axioms the following:
%
\begin{itemize}
\item $\syno(\synid(x),f) = f$, whenever $x,y \in \synOb$ and $f \in \synHom(x,y)$.
\item $\syno(f,\synid(y)) = f$, whenever $x,y \in \synOb$ and $f \in \synHom(x,y)$.
\item $\syno(\syno(f,g),h) = \syno(f,\syno(g,h))$, whenever $w,x,y,z \in \synOb$, $f \in \synHom(w,x)$, $g \in \synHom(x,y)$ and $h \in \synHom(y,z)$.
\end{itemize}

% source p1.5

A theory is presented by specifying the language and by listing the axioms.
%
The language is specified by listing the symbols and by specifying the role that each symbol plays within the language either as a sort symbol of some kind or as a particularly typed operator symbol.
%
The role that a symbol plays can always be specified by way of the assertion of a single rule.
%
In the case of a sort symbol $A$ there is a rule of the form
%
$\inferrule
  { x_1 \in \Delta_1, x_2 \in \Delta_2, \ldots\ x_n \in \Delta_n }
  { A(x_1,\ldots x_n) \isatype } $
%
that will correctly specify over what types $A$ is dependent.
%
In the case of an operator symbol $f$ a rule of the form
%
$\inferrule
  { x_1 \in \Delta_1, x_2 \in \Delta_2, \ldots\ x_n \in \Delta_n }
  { f(x_1,\ldots x_n) \in \Delta } $
%
suffices to specify of what types its arguments are to be and of what type its value will be.
%
In either case we call the symbol the \defemph{introductory rule} associated with the symbol.

For example the sort $\synHom$ of the theory of categories has introductory rule $x \in \synOb,\, y \in \synOb: \synHom(x,y) \isatype$.  The symbol $\synid$ has introductory rule $x \in \synOb : \synid(x) \in \synHom(x,x)$.

Finally, then, every theory is presented as a set of symbols each with associated introductory rule and a set of axioms.
%
And of course everything must be well formed, but we leave all that until we give the formal definition in \textsection \ref{sec:source-1-6}.

The theory of categories now looks like this:

\comment{TODO: fix formatting!}
%
\begin{theoryspec}
  $\synOb$ & $\synOb \isatype$. \\
  $\synHom$ & $x \in \synOb,\, y \in \synOb: \synHom(x,y) \isatype$. \\
  $\syno$ & $x,y,z \in \synOb,\, f \in \synHom(x,y),\, g \in \synHom(y,z) : \syno(f,g) \in \synHom(x,z)$. \\
  $\synid$ & $x \in \synOb : \synid(x) \in \synHom(x,x)$ \\
  \axioms
  \axiom{$\syno(\synid(x),f) = f$, whenever $x,y \in \synOb$ and $f \in \synHom(x,y)$.}
  \axiom{$\syno(f,\synid(y)) = f$, whenever $x,y \in \synOb$ and $f \in \synHom(x,y)$.}
  \axiom{$\syno(\syno(f,g),h) = \syno(f,\syno(g,h))$, whenever $w,x,y,z \in \synOb$, $f \in \synHom(w,x)$, $g \in \synHom(x,y)$ and $h \in \synHom(y,z)$.}
\end{theoryspec}

Whenever we speak of a model of a theory $\thU$, without qualification, then we shall mean a model in the usual sense, that is where type symbols are interpreted as sets, symbols for families of types are interpreted as families of sets, operator symbols are interpreted as operators and so on.
%
Later we shall be interpreting theories in algebraic structures, in which case type symbols will be interpreted as objects within a structure rather than as sets.

% source p1.7
\section{Examples of theories} \label{sec:source-1-2}

The first example is a theory which can be called the theory of families of elements of families of sets:

\begin{theoryspec}
  $A$ & $A$ is a type \\
  $B$ & For $x \in A$ : $B(x)$ is a type \\
  $b$ & For $x \in A$ : $b(x) \in B(x)$ \\
  \noaxioms
\end{theoryspec}

A model of this theory will consist of a set, a family indexed by this set and a distinguished element of each set in this family; which is to say that a model will consist of a set indexed family of elements of a family of sets.
%
We are not sure of the notation we should be using but if we denote the interpretation of a symbol in a model $\modM$ by that symbol superscripted by $\modM$ then a model $\modM$ consists of \comment{inline list!}(i.) a set $A^\modM$, (ii.) an $A^\modM$-indexed family of elements $b^\modM$ of the family of sets $B^\modM$.

\begin{figure}
\placeholder{TODO: Diagram!}
\caption{For every element $a$ of the set $A^\modM$ we have (i.) a set $B^\modM(a)$ and (ii.) an element $b^\modM(a)$ of the set $B^\modM(a)$.}
\end{figure}

% source p1.8

If both $\modM$ and $\modM'$ are models of this theory then a homomorphism $f : \modM \to \modM'$ consists of a function $f_A : A^\modM \to A^{\modM'}$ and an $A^\modM$-indexed family of functions $f_B$ such that for every $a \in A^\modM$, $f_B(a) : B^\modM(a) \to B^{\modM'}(f_A(a))$ and such that for every $a \in A^\modM$, $f_B(a)(b^\modM(a)) = b^{\modM'}(f_A(a))$.

Alternatively we can say that a homomorphism consists of a function $f_A : A^\modM \to A^{\modM'}$ and an operator $f_B$ such that for every $a \in A^\modM$, for every $b \in B^\modM(a)$, $f_B(a,b) \in B^{\modM'}(f_A(a))$ and satisfying $f_B(a,b^\modM(a)) = b^{\modM'}(f_A(a))$, whenever $a \in A^\modM$.
%
Now this means that there is a generalised algebraic theory whose models are just homomorphisms between the models of the given theory (in fact this is quite generally the case).
%
This theory of homomorphisms can be presented as follows:

The theory of families of elements of families of sets in the langauge $\tuple{A,B,b}$ + the same theory in the language $\tuple{A',B',b'}$ + \\
\begin{theoryspec}
  $f_A$ & For $x \in A$ : $f_A(x) \in A'$. \\
  $f_B$ & For $x \in A$, for $y \in B(x)$ : $f_B(x,y) \in B'(f_A(x))$. \\
  \oneaxiom
  \axiom{$f_B(x,b(x)) = b'(f_A(x))$, whenever $x \in A$.}
\end{theoryspec}

An example similar to the first example we call the theory of families of families of elements of families of families of sets:

\begin{theoryspec}
  $A$ & $A$ is a type \\
  $B$ & For $x \in A$ : $B(x)$ is a type \\
  $C$ & For $x \in A$, for $y \in B(x)$ : $C(x,y)$ is a type \\
  $c$ & For $x \in A$ : $c(x,y) \in C(x,y)$. \\
  % source p1.9
  \noaxioms
\end{theoryspec}

Suppose that $\modM$ is a model of this theory.
%
Then $A^\modM$ is a set.
%
For every element $a$ of the set $A^\modM$ we have a set $B^\modM(a)$ and for every element $b$ of the set $B^\modM(a)$ we have a set $C^\modM(a,b)$ and an element $c^\modM(a,b)$ of the set $C^\modM(a,b)$.

\begin{figure}
\placeholder{TODO: diagram}
\end{figure}

Now for every element $a$ of $A^\modM$, $\lambda b.C^\modM(a,b)$ is a $B^\modM$-indexed family of sets.
%
Thus $\lambda a. \lambda b. C^\modM(a,b)$, i.e.\ $c^\modM$, is an $A^\modM$-indexed family of families of sets.
%
Similarly $c^\modM$ is an $A^\modM$-indexed family of families of elements.

Note that in the presentation of this theory no harm is done if we replace the introductory rule for $C$ by the \comment{Punctuation!?} rule:--- for $x \in A$, for $y \in B(x)$ : $C(y)$ is a type, this rule having the same meaning as the given rule.
%
The expression $C(x,y)$ in the given rule depends explicitly on $x$ and $y$.
%
We say that the expression $C(y)$ in the alternative rule depends \defemph{implicitly} in $x$ by virtue of its explicit dependence on $y$ and by virtue of the dependence of $y$ on $x$.
%
In the alternative version of the theory we say that a variable has been omitted.
%
This is one way in which a theory may be informally presented.
%
We use this method and another in presenting the next theory---the theory of trees.

% source p1.10

The theory of \defemph{trees} has countably many sort symbols, no operator symbols and no axioms.
%
However, we choose to write the theory informally with just two sort symbols, one of these symbols doing the work that in a formal presentation would be shared among countably many distinct symbols.

\begin{theoryspec}
  $S_1$ & $S_1$ is a type \\
  $S$ & For $x_1 \in S_1$ : $S(x_1)$ is a type \\
  $S$ & For $x_1 \in S_1$, for $x_2 \in S(x_1)$ : $S(x_2)$ is a type \\
  \vdots & \hspace{2em} \vdots \\
  $S$ & For $x_1 \in S_1$, for $x_2 \in S(x_1)$, \ldots\ for $x_n \in S(x_{n-1})$ : $S(x_n)$ is a type \\
  \vdots & \hspace{2em} \vdots \\
  \noaxioms
\end{theoryspec} 

$S_1$, then, is a symbol denoting the set of nodes at the base of the tree.
%
If $x$ is any node of the tree then $S(x)$ is the set of nodes immediately above $x$ in the tree, that is to say the set of successor nodes to $x$.
%
In a formal presentation of this theory there would be symbols $S_1$, $S_2$, $S_3$, \ldots\ and the symbol $S_{n+1}$ would be introduced by the rule
\[ \inferrule
  {x_1 \in S_1, \\ x_2 \in S_2(x_1), \\ \ldots \\ x_n \in S_n(x_1,\ldots x_{n-1}) }
  {S_{n+1}(x_1,\ldots x_n) \isatype}
\]

We use the same methods in presenting the theory of \defemph{functors} informally.
%
The theory of functors consists of the theory of categories in the language $\tuple{\synOb,\synHom,\synid,\syno}$ + the theory of categories in the language $\tuple{\synOb,\synHom,\synid,\syno}$ (and at this point we have used the same three symbols $\synHom$, $\synid$ and $\syno$ in new roles) + \\
\begin{theoryspec}
  $\synF$ & For $x \in \synOb$ : $\synF(x) \in \synOb'$ \\
  $\synF$ & For $x,y \in \synOb$, for $f \in \synHom(x,y)$ : $\synF(f) \in \synHom(\synF(x),\synF(y))$ \\
  \axioms
  \axiom{$\synF(\synid(x)) = \synid(\synF(x))$, whenever $x \in \synOb$.}
  \axiom{$\synF(\syno(f,g)) = \syno(\synF(f),\synF(g))$, whenever $x,y,z \in \synOb$, $f \in \synHom(x,y)$ and $g \in \synHom(y,z)$.}
\end{theoryspec}

A model of this theory is just a functor.
%
The category of models is the category $\catCat^{\catTwo}$, which is to say that if $F : \catC \to \catC'$ is a functor and if $G : \catD \to \catD'$ is a functor then a homomorphism from $F$ to $G$ consists of a pair of functors $\tuple{H,H'}$ such that $H : \catC \to \catD$ and $H' : \catC' \to \catD'$ and such that
\[ \placeholder{\text{TODO: diagram!}}\]
commutes.

% source p.1.12
The final example is to indicate one way of axiomatising the disjoint union of a family of types.

If $\thU$ is a theory which includes a type symbol $A$ and a symbol $B$ for an $A$-indexed family of types then $\thU$ can be extended by three operator symbols, three axioms and one type symbol. $\synSum_A B$ in such a way that \comment{inline list!}i.\ every model of $\modM$ of $\thU$ uniquely extends to a model of the extended theory and ii.\ every model $\modM$ of the extended theory interprets the symbol $\synSum_A B$ by the set $\{ \tuple{a,b} \suchthat a \in A^\modM\ \text{and}\ b \in B^\modM(a) \}$, that is to say as the disjoint union of the family of sets interpreting $B$.
%
The extended theory is taken to be $\thU$ +

\begin{theoryspec}
  $\synSum_A B$ & $\synSum_A B$ is a type \\
  $P_1$ & For $z \in \synSum_A B$ : $P_1(z) \in A$ \\
  $P_2$ & For $z \in \synSum_A B$ : $P_2(z) \in B(P_1(z))$ \\
  $\synPr$ & For $x \in A$, for $y \in B(x)$ : $\synPr(x,y) \in \synSum_A B$ \\
  \axioms
  \axiom{$\synPr(P_1(z),P_2(z)) = z$, whenever $z \in \synSum_A B$.}
  \axiom{$P_1(\synPr(x,y)) = x$, whenever $x \in A$ and $y \in B(x)$.}
  \axiom{$P_2(\synPr(x,y)) = y$, whenever $x \in A$ and $y \in B(x)$.}
\end{theoryspec}

In future we might refer to an extension of a theory by symbols for disjoint unions.

% source p1.13
\section{Predicates as types} \label{sec:source-1-3}

\lipsum[2]

% source p1.16
\section{Essentially algebraic theories and categories with finite limits} \label{sec:source-1-4}

\lipsum[3]

% source p1.19
\section{The extra generality of the algebraic semantics} \label{sec:source-1-5}

\lipsum[4]

% source p1.22
\section{The formal definition} \label{sec:source-1-6}

\lipsum[5]

% source p1.28
\section{The substitution lemma and other lemmas} \label{sec:source-1-7}

\lipsum[6]

% source p1.39
\section{Informal syntax} \label{sec:source-1-8}

\lipsum[7]

% source p1.45
\section{Models and homomorphisms} \label{sec:source-1-9}

\lipsum[8]

% source p1.49
\section{A list of theories} \label{sec:source-1-10}

\lipsum[9]

% source p1.50
\section{Interpretations} \label{sec:source-1-11}

\lipsum[10]

% source p1.58
\section{Contexts and realisations} \label{sec:source-1-12}

\lipsum[11]

% source p1.60
\section{Intended identity of denotations} \label{sec:source-1-13}

\lipsum[12]

% source p1.69
\section{The category $\catGAT$} \label{sec:source-1-14}

\lipsum[13]

%%% Local Variables:
%%% mode: latex
%%% TeX-master: "cartmell-thesis"
%%% End: 